%%%%%%%%%%%%%%%%%%%%%%%%%%%%%%%%%%%%%%%%%
% Medium Length Graduate Curriculum Vitae
% LaTeX Template
% Version 1.1 (9/12/12)
%
% This template has been downloaded from:
% http://www.LaTeXTemplates.com
%
% Original author:
% Rensselaer Polytechnic Institute (http://www.rpi.edu/dept/arc/training/latex/resumes/)
%
% Important note:
% This template requires the res.cls file to be in the same directory as the
% .tex file. The res.cls file provides the resume style used for structuring the
% document.
%
%%%%%%%%%%%%%%%%%%%%%%%%%%%%%%%%%%%%%%%%%

%----------------------------------------------------------------------------------------
%	PACKAGES AND OTHER DOCUMENT CONFIGURATIONS
%----------------------------------------------------------------------------------------

\documentclass[margin, 10pt]{res} % Use the res.cls style, the font size can be changed to 11pt or 12pt here

\setlength{\textwidth}{5.1in} % Text width of the document


%\usepackage{helvet} % Default font is the helvetica postscript font
%\usepackage{newcent} % To change the default font to the new century schoolbook postscript font uncomment this line and comment the one above
\usepackage{tgpagella}
\usepackage{enumitem}
\usepackage[colorlinks=true, urlcolor=Blue]{hyperref}% \usepackage[hidelinks=true]{hyperref}
\usepackage{fancyhdr} \pagestyle{fancy} \renewcommand{\headrulewidth}{0pt}
\usepackage{dirtytalk}
\usepackage[usenames, dvipsnames]{color}
\usepackage{calc}


%----------------------------------------------------------------------------------------
%	CUSTOM COMMANDS
%----------------------------------------------------------------------------------------

\newcommand{\Year}[1]{\emph{#1}}
\newcommand{\uni}[1]{\emph{#1}}
\newcommand{\eventYear}[2]{\uni{#1}\hfill{\Year{#2}}\\}

% Color the section headers
\definecolor{crimson}{RGB}{164, 16, 52}
\let\oldsection\section
\renewcommand{\section}[1]{\oldsection{\textcolor{crimson}{#1}}}

% make "C++" look pretty when used in text by touching up the plus signs
\newcommand{\CPP}
{C\nolinebreak[4]\hspace{-.05em}\raisebox{.22ex}{\footnotesize\bf ++}}


\author{Arya D. McCarthy}
\date{\oldstylenums{\today}}

% Headers
\makeatletter
\lfoot{\@author}
\cfoot{\@date}
\makeatother
\rfoot{\oldstylenums{\thepage}}

% Hrefs
\newcommand{\JHU}{\href{https://www.jhu.edu/}{Johns Hopkins University}}
\newcommand{\jhu}{\href{https://www.jhu.edu/}{JHU}}
\newcommand{\SMU}{\href{https://www.smu.edu/}{Southern Methodist University}}
\newcommand{\smu}{\href{https://www.smu.edu/}{SMU}}
\newcommand{\HPHS}{\href{https://hs.hpisd.org}{Highland Park High School}}
\newcommand{\STAN}{\href{https://www.stanford.edu/}{Stanford University}}
\newcommand{\UOE}{\href{https://www.ed.ac.uk/}{University of Edinburgh}}
\newcommand{\clsp}{\href{https://www.clsp.jhu.edu}{CLSP}}
\newcommand{\CLSP}{\href{https://www.clsp.jhu.edu}{Center for Language and Speech Processing}}


\renewcommand{\oldstylenums}[1]{{\fontfamily{pplj}\selectfont #1}}

\hyphenation{SIG-MOR-PHON}

\begin{document}

%----------------------------------------------------------------------------------------
%	NAME AND ADDRESS SECTION
%----------------------------------------------------------------------------------------

%\noindent
\begin{minipage}{0.6180339887498949\textwidth}
\begin{flushright}
  \CLSP\\
  \href{https://www.jhu.edu/}{Johns Hopkins University}\\
  %Hackerman 321\\
  %3400 North Charles Street\\
  Baltimore, MD% 21218, USA
  \end{flushright}
\end{minipage}% This must go next to `\end{minipage}`
\begin{minipage}{0.3819660112501051\textwidth}
  \begin{flushright}
      Phone: \href{tel:+14698343984}{\textsc{(469) 834-3984}}\\
      \href{mailto:arya@jhu.edu}{\textsf{arya@jhu.edu}}\\
      \href{https://cs.jhu.edu/~arya}{\textsf{cs.jhu.edu/{\raise.17ex\hbox{$\scriptstyle\mathtt{\sim}$}}arya}}
  \end{flushright}
\end{minipage}

\makeatletter
\moveleft2.17\hoffset\centerline{\textsc{\LARGE \@author}} % Your name at the top
\makeatother
 \vspace{-0.5em}
\moveleft\hoffset\vbox{\hrule width\resumewidth height 0.5pt}\smallskip % Horizontal line after name; adjust line thickness by changing the '1pt'
 \vspace{-2em}
\begin{resume}

%----------------------------------------------------------------------------------------
%	CONTACT INFORMATION
%----------------------------------------------------------------------------------------

\section{Research Focus}

I work on \textbf{morphology} as a tool for \textbf{Machine Translation}. The effort is part of the DARPA Low Resource Languages for Emergent Incidents (LORELEI) project, which focuses on \textbf{rapid development} of NLP tools for languages with little or no training data. This includes building a \textbf{massively multilingual} annotated lexicon for the Universal Morphology (UniMorph) project, as well as developing \textbf{adaptation} methods for existing translation models.

\vspace{-0.6em}
Recently, I have also worked on deep, neural models of sensory perception in PyTorch. 

\vspace{-0.6em}
I also remain active in the \textbf{network science} community, where I specialize in \textbf{community detection}.



%----------------------------------------------------------------------------------------
%	EDUCATION SECTION
%----------------------------------------------------------------------------------------

\section{Education}

\eventYear{\JHU}{}
	\oldstylenums{2017} -- Present. Ph.D. in Computer Science
	\begin{description}[noitemsep, labelindent=\widthof{2017}]
%	\item[Affiliation:] \CLSP
	\item Advisor: \href{https://www.cs.jhu.edu/~yarowsky/}{David Yarowsky}
	\end{description}

\eventYear{\SMU}{}
	\oldstylenums{2017}. M.S. in Computer Science
	\begin{description}[noitemsep, labelindent=\widthof{2017}]
	\item Thesis: \emph{\href{https://search.proquest.com/docview/1907180434}{The Leximin Method for Hierarchical Community Detection}}
	\item Advisor:  \href{https://lyle.smu.edu/~matula/}{David W. Matula}
	\end{description}%
	\vspace{-1.2em}
	\oldstylenums{2017}. B.S. in Computer Science, Mathematics
	\begin{description}[noitemsep, labelindent=\widthof{2017}]
	\item Magna Cum Laude; \href{https://www.smu.edu/dedman/studentresources/universityhonors}{Honors in the Liberal Arts, University Honors Program}
	\end{description}
	
%\eventYear{\HPHS}{2013}
%	Valedictorian
	
\emph{Single terms of study at the \UOE{} (\oldstylenums{2015}) and \STAN{} (\oldstylenums{2014})}

%----------------------------------------------------------------------------------------
%	EMPLOYMENT SECTION
%----------------------------------------------------------------------------------------


\section{Employment}

\eventYear{\href{https://hltcoe.jhu.edu}{Human Language Technology Center of Excellence}\emph{, Baltimore, MD}}{}
Summer \oldstylenums{2018}. Participant, \href{https://hltcoe.jhu.edu/research/scale/scale-2018/}{Summer Camp for Applied Language Exploration (SCALE)}
\begin{description}[nosep, labelindent=1em]
\item Supervisor: \href{https://www.cs.jhu.edu/~kevinduh/}{Kevin Duh}
\item Area: Domain adaptation for machine translation
\end{description}

\eventYear{\href{https://www.smu.edu/Lyle/Institutes/DeasonInstitute}{Darwin Deason Institute for Cyber Security}\emph{, Dallas, TX}}{}
\oldstylenums{2015} -- \oldstylenums{2016}. Research Assistant
\begin{description}[nosep, labelindent=1em]
\item Supervisor: \href{https://s2.smu.edu/~mitch/}{Mitch Thornton}
\end{description}

%----------------------------------------------------------------------------------------
%	GRANTS SECTION
%----------------------------------------------------------------------------------------

\section{Grants and \\ Awards}

\setlist[description]{font=\normalfont}

\begin{description}

\item[\oldstylenums{2017}] Dean's Award: Best CSE poster at \href{https://www.smu.edu/graduate/CurrentStudents/ResearchDay}{SMU Research Day \oldstylenums{2017}}. \emph{AirWare: In-air gesture recognition using ultrasonic Doppler signatures and deep neural networks} %
\\ \href{https://sites.smu.edu/des/registrar/honorawards/?a=awards&year=2017}{Charles J. Pipes Award for Outstanding Performance in Mathematics}%

\item[\oldstylenums{2016}] 
 \href{https://www.smu.edu/Dedman/DCII/Programs/Mayer}{Robert Mayer Interdisciplinary Fellowship}%
 \\ \href{https://www.smu.edu/Dedman/DCII/Programs/Hamilton}{Hamilton Undergraduate Research Fellowship}%
 \\ \href{https://www.smu.edu/EnrollmentServices/Registrar/AcademicCeremonies/HonorsConvocation/HonorSocieties}{Robert S. Hyer Society}. Highest academic honor at SMU\@.
\item[\oldstylenums{2015}] \href{https://uraf.harvard.edu/amgen-scholars}{Harvard--Amgen Scholarship}; host \href{https://www.eecs.harvard.edu/shieber/}{Stuart Shieber}%
\\ Upsilon Pi Epsilon
\item[\oldstylenums{2014}] Tau Beta Pi
\item[\oldstylenums{2013}] \href{https://www.smu.edu/Academics/PS}{President's Scholarship}. Highest merit scholarship at SMU\@. One of 21 in class \hphantom{m}of \oldstylenums{2017}.
\\ Campus Community Award: Full room and board for 4 years, awarded for \hphantom{m}leadership on campus.
\end{description}

%\textbf{\href{http://www.smu.edu/Dedman/DCII/Programs/Mayer}{Robert Mayer Interdisciplinary Fellowship}} \\
%\begin{description}[nosep, labelindent=1em]
%\item[Awarding body:] \href{http://www.smu.edu/Dedman/DCII}{SMU Dedman College Interdisciplinary Institute}
%\item[Amount:] \$1,500
%\item[Supervisor:] \href{http://faculty.smu.edu/snorris/}{Scott Norris, Ph.D.} and \href{https://www.smu.edu/News/Experts/Matthew-Wilson}{Matthew Wilson, Ph.D.}
%\item Coming Together, Mathematically: Dynamical Models for Increased Uniformity and Polarization in American Politics.
%\end{description}
%\textbf{\href{http://www.smu.edu/Dedman/DCII/Programs/Hamilton}{Hamilton Undergraduate Research Fellowship}} \\
%\begin{description}[nosep, labelindent=1em]
%\item[Awarding body:] \SMU
%\item[Amount:] \$3,000
%\item[Supervisor:] \href{http://faculty.smu.edu/snorris/}{Scott Norris, Ph.D.}
%\item Implemented and extended waveform relaxation solver for modeling interrelated differential equations without loss of accuracy.
%\end{description}
%\textbf{\href{https://uraf.harvard.edu/amgen-scholars}{Harvard--Amgen Scholarship}}
%\begin{description}[nosep, labelindent=1em]
%\item[Awarding body:] \href{https://uraf.harvard.edu/home}{Harvard University}; \href{http://www.amgen.com/responsibility/amgen-foundation/}{Amgen Foundation}
%\item[Amount:] \$4,500
%\item[Supervisor:] \href{http://www.eecs.harvard.edu/shieber/}{Stuart Shieber, Ph.D.}
%\item Toward coreference resolution shared task, adapted multithreaded feature extraction code in Java to serialize and deliver features to Torch neural network.
%\end{description}


%\eventYear{\SMU}{}
%\href{https://datascience.smu.edu/academics/curriculum/coursedescriptions/#data-science}{Doing Data Science (MSDS 6306)} \hfill \emph{Spr., Sum., Fall 2018}
%\begin{description}[noitemsep, labelindent=1em]
%\item[Role:] Teaching Assistant
%\item[Professor:] Faizan Javed
%\item For two sections, I assisted weekly synchronous sessions and graded reports in SAS, Python, and R. Part of SMU's Master's of Science in Data Science program.
%\end{description}
%
%\filbreak
%
%\href{https://datascience.smu.edu/academics/curriculum/coursedescriptions/#quantifying}{Quantifying the World (MSDS 7333)}\hfill \emph{Fall 2017}
%\begin{description}[noitemsep, labelindent=1em]
%\item[Role:] Grader
%\item[Professor:] Owen Martin, John Verostek
%\end{description}
%\vspace{-1.2em}
%\href{https://tylermoore.ens.utulsa.edu/courses/cse3353/}{Fundamentals of Algorithms (CSE 3353)} \hfill \emph{Fall 2014}
%\begin{description}[noitemsep, labelindent=1em]
%\item[Role:] Teaching Assistant
%\item[Professor:] \href{https://tylermoore.ens.utulsa.edu}{Tyler Moore, Ph.D.}
%\item Led 2 office hours weekly, answered questions via Piazza, and graded all assignments and exams for 30 students.
%\end{description}



%----------------------------------------------------------------------------------------
%	AWARDS SECTION
%----------------------------------------------------------------------------------------

%\section{Awards}

 %----------------------------------------------------------------------------------------
%	PUBLICATIONS SECTION
%----------------------------------------------------------------------------------------

\section{Publications}
\vspace{-0.1em}
%\paragraph{In Progress}
%\begin{description}
%\item \textbf{Arya D. McCarthy} and Ryan Cotterell. \emph{Deep Berlin and Kay: A latent-variable model of languages' color inventories}. {Transactions of the ACL \oldstylenums{2019}}.
%\item \textbf{Arya D. McCarthy}, Adi Renduchintala, and Ryan Cotterell. \emph{A principled approach to morphological inflection}. Proceedings of NAACL \oldstylenums{2019}.
%\item \textbf{Arya D. McCarthy}, Tongfei Chen, Rachel Rudinger, and David W. Matula. \emph{Metrics matter in community detection}. Applied Network Science \oldstylenums{2019}.
%\end{description}

%\paragraph{Refereed Journal Articles}
%\begin{enumerate}
%\item Nibhrat Lohia, Raunak Mundada, \textbf{Arya D. McCarthy}, and Eric C. Larson. \href{http://cs.jhu.edu/~arya/lohia+al.tochi2018.draft.pdf}{\emph{AirWare: Utilizing embedded audio and infrared signals for in-air hand-gesture recognition}}. {Transactions on Computer-Human Interaction \oldstylenums{2018}}. \emph{(Under review)}
%\end{enumerate}

\paragraph{Refereed Journal Papers}
\begin{enumerate}[resume]
\item \textbf{Arya D. McCarthy} and Ryan Cotterell. \emph{A deep, latent-variable model for languages' color inventories}. TACL. \emph{(Under review)}
\end{enumerate}

\paragraph{Refereed Conference Submissions}
\begin{enumerate}[resume]
\item \textbf{Arya D. McCarthy}. \href{https://cs.jhu.edu/~arya/mccarthy.iccna18.pdf}{\emph{An exact No Free Lunch theorem for clustering and community detection}}.  Complex Networks (ICCNA) \oldstylenums{2018}. \emph{(To appear)}

\item \textbf{Arya D. McCarthy} and David W. Matula. \href{https://cs.jhu.edu/~arya/mccarthy+matula.iccna18.pdf}{\emph{Evaluating the leximin method for community detection}}. Complex Networks (ICCNA) \oldstylenums{2018}. \emph{(To appear)}

\item Brian Thompson, Huda Khayrallah, Antonios Anastasopoulos, \textbf{Arya D. McCarthy}, Kevin Duh, Rebecca Marvin, Paul McNamee, Jeremy Gwinnup, Tim Anderson and Philipp Koehn. \href{https://www.aclweb.org/anthology/W18-6313}{\emph{Freezing subnetworks to analyze domain adaptation in neural machine translation}}. Proceedings of WMT \oldstylenums{2018}.

\item Christo Kirov, Ryan Cotterell, John Sylak-Glassman, G\'eraldine Walther, Ekaterina Vylomova, Patrick Xia, Manaal Faruqui, \textbf{Arya D. McCarthy}, Sandra K{\"u}bler, David Yarowsky, Jason Eisner, and Mans Hulden. \href{http://www.lrec-conf.org/proceedings/lrec2018/pdf/789.pdf}{\emph{UniMorph \oldstylenums{2.0}: Universal morphology}}. Proceedings of LREC \oldstylenums{2018}.
\end{enumerate}

\paragraph{Refereed Workshop Proceedings}
\begin{enumerate}[resume]
\item \textbf{Arya D. McCarthy}, Miikka Silfverberg, Ryan Cotterell, Mans Hulden, and David Yarowsky. \href{http://www.aclweb.org/anthology/W18-6011}{\emph{Marrying Universal Dependencies and Universal Morphology}}. Proceedings of EMNLP UDW \oldstylenums{2018}.

\item \textbf{Arya D. McCarthy} and David W. Matula.  \href{https://cs.jhu.edu/~arya/mccarthy+matula.ns18.pdf}{\emph{Normalized mutual information exaggerates community detection performance}}. SIAM Workshop on Network Science \oldstylenums{2018}.
\end{enumerate}


\paragraph{Invited Publications}
\begin{enumerate}[resume]
\item Ryan Cotterell, Christo Kirov, John Sylak-Glassman, G\'{e}raldine Walther, Ekaterina Vylomova, \textbf{Arya D. McCarthy}, Katharina Kann, Sebastian Mielke, Garrett Nicolai, Miikka Silfverberg, David Yarowsky, Jason Eisner, and Mans Hulden. \href{https://aclweb.org/anthology/K18-3001}{\emph{The CoNLL--SIGMORPHON \oldstylenums{2018} shared task: Universal morphological reinflection}}. Proceedings of CoNLL--SIG\-MOR\-PHON \oldstylenums{2018}.
\end{enumerate}

\paragraph{Non-Public Technical Reports}
\begin{enumerate}[resume]
\item \textbf{Arya D. McCarthy}. \emph{Design and Implementation of a Method of Abstractly Simulating Cyber Security Vulnerabilities: Embedded Markov and Discrete Event Simulation Approaches.} Deason Institute for Cyber Security \oldstylenums{2016}.
\end{enumerate}

%\textbf{Non-Refereed}
%\begin{enumerate}
%\item \textbf{Arya D. McCarthy}, Scott A. Norris. \emph{Toward Fast, Accurate Simulation of Gap Junctions in NNs.} National Collegiate Research Conference \oldstylenums{2017}. \emph{(Abstract)}
%\item \textbf{Arya D. McCarthy}, Sam Wiseman, Stuart Shieber. \emph{Toward Multilingual Neural Coreference Resolution.} Harvard--Amgen Scholars Program \oldstylenums{2015}. \emph{(Abstract)}
%\end{enumerate}


%----------------------------------------------------------------------------------------
%	TEACHING SECTION
%----------------------------------------------------------------------------------------

\section{Teaching}

\begin{description}
\item[\oldstylenums{2018}] Teaching Assistant, Natural Language Processing, \href{https://www.cs.jhu.edu/~jason/}{Jason Eisner}, JHU, Fall.%
\\ Teaching Assistant, Doing Data Science, Faizan Javed, SMU MSDS, Spr, Sum, \hphantom{m}Fall.
\item [\oldstylenums{2017}] Grader, Quantifying the World, Owen Martin \& John Verostek, SMU MSDS, \hphantom{m}Fall.\\
	Guest Instructor, Fundamentals of Algorithms, Vidroja Debroy, Spring.
\item [\oldstylenums{2015}] Teaching Assistant, Fundamentals of Algorithms, Tyler Moore, Spring.
\end{description}


%----------------------------------------------------------------------------------------
%	INVITED TALKS SECTION
%----------------------------------------------------------------------------------------

\section{Invited Talks}
\begin{enumerate}
\item Coming Together, Mathematically: Dynamical Models for
 Increased Uniformity and Polarization in American Politics. May \oldstylenums{2017}\\
Location: Southern Methodist University (Hamilton Fellows Series)
%Host: Kathleen Hugley-Cook
\item Toward Fast, Accurate Simulation of Gap Junctions in NNs. March \oldstylenums{2017}\\
Location: Southern Methodist University (as Summer Research Fellow)
%Host: Robert Kehoe
\end{enumerate}

 %----------------------------------------------------------------------------------------
%	SERVICE SECTION
%----------------------------------------------------------------------------------------

\section{Service}

\begin{description}
\item Shared task organizer: CoNLL--SIGMORPHON \oldstylenums{2018} (Universal Morphological Reinflection)
\item Program committee: SIGMORPHON, WMT
\item Reviewer for: WMT (\oldstylenums{2018}), SIGMORPHON (\oldstylenums{2018}), EMNLP (\oldstylenums{2018} secondary)
\item Open-source contributions: \href{https://networkx.github.io}{\texttt{networkx}}, \texttt{scikit-learn}, \texttt{PyTorch/tutorials}
%\item Webmaster: ACL SIGMORPHON, UniMorph morphology project, SMU Ubiquitous Computing Lab (\oldstylenums{2016} -- \oldstylenums{2017})
\item Judge for ACM-ICPC contest at JHU (\oldstylenums{2017, 2018})
\item Diversity and Inclusion committee for Department of Computer Science (\oldstylenums{2017 --} )
%\item Founder and editor-in-chief, \emph{SMU Journal of Undergraduate Research} (\oldstylenums{2014} -- \oldstylenums{2017})
%\item Editor-in-chief, \emph{Kairos} interdisciplinary magazine  (\oldstylenums{2015} -- \oldstylenums{2016})
%\item Vice-President, SMU Tau Beta Pi  (\oldstylenums{2016} -- \oldstylenums{2017})
%\item President, SMU Upsilon Pi Epsilon  (\oldstylenums{2015} -- \oldstylenums{2017})
%\item Sole student member of SMU Undergraduate Research Steering Committee,  (\oldstylenums{2015} -- \oldstylenums{2017})
%\item Common Reading Selection Committee, SMU (\oldstylenums{2015})
\end{description}

%\textbf{Shared Task Organizer}
%
%\begin{itemize}
%\item \href{https://sigmorphon.github.io/sharedtasks/2018/}{CoNLL--SIGMORPHON 2018 Shared Task: Cross-Lingual Morphological Reinflection}
%
%\end{itemize}
%
%\textbf{Professional Service}
%
%\begin{itemize}[noitemsep]
%\item Program Committee,  \href{https://sigmorphon.github.io}{SIGMORPHON} (\href{https://sigmorphon.github.io/workshops/2018/}{2018})
%\item Reviewer: WMT (\href{http://www.statmt.org/wmt18/}{2018}), EMNLP (2018 secondary)
%\item Open-source contributions: \href{http://networkx.github.io}{\texttt{networkx}}, \texttt{PyTorch/tutorials}, \texttt{scikit-learn}
%%\item Webmaster: \href{https://sigmorphon.github.io}{ACL SIGMORPHON}, \href{https://unimorph.github.io}{UniMorph morphology project}, \href{http://ubicomp.lyle.smu.edu}{SMU Ubiquitous Computing Lab} (2016--2017)
%\item Founder and editor-in-chief, \href{https://smuresearch.wordpress.com/journal/}{\emph{SMU Journal of Undergraduate Research}} (2014--2017)
%\item Editor-in-chief, \href{https://smuresearch.wordpress.com/kairos/}{\emph{Kairos}} interdisciplinary magazine (2015--2016)
%\end{itemize}

% \subsection{Community Service}

%\textbf{University Service}

%\begin{itemize}[noitemsep]
%\item Judge for \href{https://en.wikipedia.org/wiki/ACM_International_Collegiate_Programming_Contest}{ACM-ICPC} contest at JHU
%\item Diversity and Inclusion Committee for \href{https://www.cs.jhu.edu}{Department of Computer Science}
%\item Faculty Liaison Committee for \clsp
%\item Vice-President, SMU Tau Beta Pi (2016--2017)
%\item President, SMU Upsilon Pi Epsilon (2015--2017)
%\item Sole student member of SMU Undergraduate Research Steering Committee
%\item Common Reading selection committee, SMU
%\end{itemize}

%----------------------------------------------------------------------------------------
%	REFERENCES SECTION
%----------------------------------------------------------------------------------------

%\section{References}
%
%\begin{description}[noitemsep]
%\item[David Yarowsky] (\href{mailto:yarowsky@jhu.edu}{yarowsky@jhu.edu}), \JHU
%\item[Ryan Cotterell] (\href{mailto:rcotter2@jhu.edu}{rcotter2@jhu.edu}), Cambridge University
%\item[David W. Matula] (\href{mailto:matula@smu.edu}{matula@smu.edu}), \SMU
%\item[Scott A. Norris] (\href{mailto:snorris@smu.edu}{snorris@smu.edu}), \SMU
%\item[Eric C. Larson] (\href{mailto:eclarson@smu.edu}{eclarson@smu.edu}), \SMU
%\item[Daniel W. Engels] (\href{mailto:dwe@alum.mit.edu}{dwe@alum.mit.edu}), \SMU
%\end{description}

%----------------------------------------------------------------------------------------
%	SKILLS SECTION
%----------------------------------------------------------------------------------------

%\section{Misc}
%
%\begin{description}
%\item[Formal Languages:] Python, R, SAS, \CPP, SQL, Objective-C, Java, JavaScript, MATLAB%, \LaTeX
%\item[Machine Learning Frameworks:] PyTorch, scikit-learn, TensorFlow
%\item[Natural Languages:] English, Farsi, Spanish, Italian, Romanian (written)%; basic Dutch and German
%\item[Graduate-Level Coursework in Computer Science:] Linguistic \& Sequence Modeling, Machine Translation, Deep Learning, Machine Learning in Python, Data Mining, Algorithm Engineering, Computer Architecture
%\item[Graduate-Level Coursework in Mathematics and Statistics:] Bayesian Statistics, Linear Programming, Data Science, Numerical Methods I (numerical linear algebra) and II (numerical analysis), Mathematical Models in Biology
%\item[Graduate-Level Coursework in Linguistics:] Language and Thought
%\end{description}

\section{Extracurricular Activities} 
\begin{description}
\item Graduate president of  Ballroom Dance @ JHU
\item Bagpiper
\end{description}

%----------------------------------------------------------------------------------------
%	EXTRA-CURRICULAR ACTIVITIES SECTION
%----------------------------------------------------------------------------------------


%\section{Extracurricular Activities} 

%Bagpiper 

%Ballroom Dancer% \\
%If my first NLP paper is accepted to EMNLP 2018, I \emph{will} wear my kilt.

%----------------------------------------------------------------------------------------

\end{resume}

\end{document}